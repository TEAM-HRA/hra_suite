\section{Domena częstotliwości - analiza spektralna}

Drugą grupą technik wykorzystywanych do analizy rytmu serca jest analiza
spektralna. Należy zaznaczyć że większa część teoretycznego opisu HRV wywodzi
się właśnie z zastosowania takiego rodzaju technik \cite{task2, hrv_origins, dynamicMalik}. 
W analizie spektralnej, podobnie jak w domenie czasowej, jest brana pod uwagę wariancja
lecz stosuje się inne nazewnictwo oraz troche inny język matematyczny.
Mianowicie wariancja przy podejściu spektralnym nazywana jest mocą spektralną, która jest 
wyliczana w podziale na częstotliwości z badanego zakresu, przy czym podział zakresu
częstotliwości jest dokonywany trochę sztucznie bez silnego związku z fizjologią. Podział
zakresu pasma dla krótkich nagrań, rzędu 2-5 minut, zwykłe przebiega następująco: (a) 
zakres 0,15 - 0,4 Hz wysoka częstotliwość, (b) zakres 0,04 - 0,15 Hz niska częstotliwość, (c)
poniżej 0,04 bardzo niska częstotliwość, dodatkowo (d) zakres poniżej 0,003 Hz,
występujący w przypadku dłuższych nagrań np. 24 godzinnych, zwany ultraniską częstotliwością.
Przyjmuje się że wplywy układu współczulnego oraz oddychania \cite{task2, hrv_origins, dynamicMalik, Hainsworth} są reprezentowane przez widmo w zakresie wysokich częstotliwości; na widmo z
zakresu niskich częstotliwości wpływają oba podukłady autonomicznego układu nerwowego; 
natomiast dla widma z pozostałych zakresów tzn bardzo niskich i ultra niskich częstotliwości
nie jest znany, w obecnym stanie wiedzy, związek z fizjologią w kontekście HRV. 

Wykorzystywana w analizie spektralnej funkcja gęstości spektralnej jest obliczana wprost i
można tego dokonać w dwojaki sposób: z wykorzystaniem metod nieparametrycznych albo przy
pomocy estymujących metod parametrycznych dla których wymagane jest przyjęcie $a\:priori$
pewnego modulu matematycznego opisującego sygnał wejściowy uzyskany z szeregu $RR$ a
wynikiem tej estymacji jest periodogram \cite{task2, hrv_origins, shumway, lomb_o, scargle, lombCasti, dsp_engel}. Wyznaczanie wariancji odpowiadającej pasmu
częstotliwości dokonuje się przy pomocy całkowania po częstotliwościach związanych z tym
pasmem. 

W praktyce, wśród metod nieparametrycznych szczególnym uznaniem i zastosowaniem cieszy się
metoda zwana szybką transformatą Fouriera (ang. Fast Fourier Transform, FFT). Wsród metod
parametrycznych najpopularniejsza jest metoda autoregresyjna wykorzystująca do estymacji
wielomian, czego konsekwencją jest to że nie wszystkie dane są ujęte w periodogramie, tym
niemniej ciągle jest możliwe wnioskowanie statystyczne \cite{task, hrv_origins, dsp_engel, lomb,scargle,lombCasti,shumway}. Cechą wspólną obu metod jest
konieczność użycia resamplingu lub inaczej interpolacji tachogramu przez równomierne w
czasie próbkowanie a także potrzeba interpolacji 'zakłóceń' tzn. uderzeń ektopowych oraz
artefaktów. Resampling tachogramu zwykle wykonuje z częstością co 250 ms.

Jednym z większych problemów przy analizie spektralnej szeregu czasowego $RR$ jest jego
niestacjonarność. Objawia się to między innymi tym że skład widmowy spektralnej funkcji
gęstości nie odpowiada oscylacjom w systemie sercowo-naczyniowym, na przykład daje się
zaoobserwować duży wkład powolnych trędów czy też brzegowych własności szeregu czasowego
w widmie niskich częstotliwości. Próbą poradzenia sobie z tymi problemami jest użycie metod
pozwalających na niwelowanie niestacjonarności czy też gwałtownym zmianom fazy, jednakże
jest to związane z niepożądaną zmianą struktury szeregu czasowego. W związku z tym wydaje
się zasadne użycie innych specjalistycznych metod oscylacyjnych, przykłady można znaleść
w pracach \cite{task2, hrv_origins, prsa_k}.     
 
Parametry analizy spektralnej jako takie bardzo różnią się pomiędzy osobami, zatem w celu
umożliwienia ich porównywania stosuje się tak zwane znormalizowane parametry analizy
spektralnej, LFnu oraz HFnu, wyrażone przez następujące relacje:
\begin{equation}
\mathrm{HFnu=\frac{HF}{HF+LF}},\qquad \mathrm{LFnu=\frac{LF}{HF+LF}}. \label{HFLFnu}
\end{equation}
gdzie: HF - moc widma szeregu $RR$ w zakresie wysokich częstotliwości 0,15-0,4 Hz;
LF - moc widma szeregu $RR$ w zakresie niskich częstotliwości 0,04-0,15 Hz.

\subsection{Transformata Fouriera}

W ogólnym zarysie transformata Foriera polega na rozkładzie sygnału na szereg funkcji
sinusidalnych z odpowiednio dobranymi aplitudami i częstotliwościami. Istnieją cztery typy
transformat \cite{dsp} które dzielą się ze względu na: (a) zachowanie sygnału - periodyczny,
aperiodyczny; (b) typ sygnału - ciągły, dyskretny. Wspólną ich cechą jest to że
zawsze operują na sygnałach nieskończonych, zatem sygnał uzyskany z ograniczonego w czasie
szeregu $RR$ nie jest poprawnym, z formalnego punktu widzenia, źródłem danych dla
transformaty Fouriera. Najcześciej tą niedogodność rozwiązuje się na dwa sposoby:
(a) przyjmuje się wartości zero z lewej i prawej strony sygnału, zatem uzyskuje się sygnał
dyskretny aperiodyczny; lub (b) zakłada się występowania kopii sygnału z lewej i prawej jego
strony, zatem uzyskuje się sygnał dyskretny periodyczny. Z powodów praktycznych
lepsze do obliczeń numerycznych jest podejścia (b) gdyż nie ma problemu z koniecznością
reprezentowania aperiodycznego sygnału przez nieskończoną liczbę sinusoid, czyli w tym
przypadku stosuje się tzw dyskretną transformatę Fouriera (ang. DFT - Discrete Fourier
Transform). Działanie DFT na $N$-elementowy dyskretny sygnał czasowy generuje dwa $N/2+1$
elementowe szeregi amplitud $C$ oraz $S$ dla funkcji \emph{cosinus} oraz \emph{sinus}, co
wyraża następująca relacja:

\begin{equation}
  x[i] = \sum_{k=0}^{N/2} \bar{C}[k]\cos( \frac{2\pi k i}{N}) + \sum_{k=0}^{N/2}\bar{S}[k]\sin( \frac{2\pi k i}{N})
  \label{eq:dft}
\end{equation}
gdzie $x$ to szereg z domeny czasowej, $\bar{C}[k] = \frac{C[k]}{N/2}$, $\bar{S}[k] = -\frac{S[k]}{N/2}$, $\bar{C}[0] = \frac{C[0]}{N}$, $\bar{C}[N/2] = \frac{C[N/2]}{N}$ a $i=[0,\ldots,N]$. Indeks $k$ reprezentuje
częstotliwości funkcji \emph{cosinus} oraz \emph{sinus}, a $C[k]$ i $S[k]$ opisują
amplitudę, przypadającą na daną częstotliwość. Postać wzoru (\ref{eq:dft}) określa także
sposób przejścia (syntezy) sygnału z domeny częstotliwości do domeny czasowej.
Jadnakże należy zwrócić uwagę na bardzo ważny fakt związany z analizą szeregów czasowych
$RR$. Pełna transformata Fouriera zawiera dwa człony:
amplitudowe widmo mocy które odpowiada wariancji z domeny czasowej oraz widmo fazowe
które nie jest rozpatrywane w obszarze HRV. Fakt odrzucania członu fazowego powoduje że
nie można przejść z domeny częstotliwości do domeny czasowej i dotyczy to specyficznie
szeregów czasowych $RR$, a nie ogólnych matematycznych rozważań w kontekście relacji (\ref{eq:dft}). 
Powyższe zdanie można wyrazić obrazowo w ten sposób: nie jest możliwe odtworzenie kształtu
szeregu $RR$, w szczególności serii zwolnień i przyspieszeń, wychodząc z domeny
częstotliwości do domeny czasowej z powodu 'zgubienia' widma fazowego w transformacie
Fouriera.

Użycie DFT do analizy spektralnej HRV, wymaga aby wejściowy sygnał bedący szeregiem odstępów
$RR$ przedstawić w postaci tachogramu o równoodległych punktach \cite{splines}. Tachogram jest następnie
interpolowany oraz wygładzany przy pomocy odpowiedniego filtra np typu \emph{boxcar} \cite{boxcar}.

Spośród kilku metod przy pomocy których można wykonać analizę spektralną DFT,
algorytm szybkiej transformaty Fouriera (ang. fast Fourier transform, FFT) \cite{cooleyfft} cieszy się
największym uznaniem z powodu najlepszej wydajności obliczeniowej.

\subsection{Periodogram Lomba}

Wśród metod parametrycznych możemy wyróżnić metodę będącą rozwinięciem analizy spektralnej,
znaną pod nazwą periodogramu Lomba-Scargle'a (LSP). Szczególna użyteczność tej metody
polega między innymi na tym że nie wymaga resamplingu, tak jak wcześniej wspomniana 
metoda autoregresyjna. Jest ona realizowana przy pomocy metody najmniejszych kwadratów.
Ze względu na brak resamplingu nie jest konieczne aby szereg czasowy, np. $RR$, miał stałą
częstość próbkowania, zatem można użyć bezpośrednio oryginalnych danych \cite{scargle, lomb_o, c++, thong},
a wszelkie pobudzenia ektopowe, zwane inaczej pobudzeniami przeniesionymi oraz
artefakty techniczne nie wpływają na poprawność działania periodogramu Lomba-Scargle'a.

Pochodna powyższej metody o nazwie uśrednionego periodogramu Lomba-Scargle'a została
opracowana w trakcie eksperymentu w którym osoby biorące w nim udział miały za zadanie
oddychać z zadaną częstotliwością. W eksperymencie badacze weryfikowali
hipotezę czy istnieje sprzężenie mechaniczne pomiędzy czynnością układu sercowo-naczyniowego
a oddechem o określonym rytmie. 

Technicznie metoda uśrednionego periodogramu Lomba-Scargle'a (ALSP) wykorzystuje uśredniony i
znormalizowany periodogram Lomba-Scargle'a dla badanej populacji szeregów. Wyrażanie
definiujące znormalizowany periodogram z odjętą średnią dla jednego szeregu z populacji
jest następujące \cite{c++, thong, vanicek}:

\begin{eqnarray}
P_{X}(f)&=&\frac{1}{2SDNN^{2}}\left\{\frac{\left[ \sum_{i=1}^{n}(RR(t_{i})-\overline{RR})\cos(2\pi f(t_{i}-\tau))\right]^{2}}{\sum_{i=1}^{n}\cos^{2}(2\pi f(t_{i}-\tau))}\right. \label{LSPeriodogram}\nonumber\\ 
&+&\left. \frac{\left[ \sum_{i=1}^{n}(RR(t_{i})-\overline{RR})\sin(2\pi f(t_{i}-\tau))\right]^{2}}{\sum_{i=1}^{n}\sin^{2}(2\pi f(t_{i}-\tau))}\right\}
\label{eq:lomb} ,
\end{eqnarray}
gdzie $n$ jest długością szeregu czasowego, wyrażoną w~punktach, a~nie w~czasie,
$\overline{RR}$ jest jego średnią, $SDNN^{2}$ jego wariancją, a~$\tau(t)$ jest zależnym od
częstości przesunięciem fazowym sprawiającym, że periodogram jest niezmienniczy względem
przesunięć w czasie \cite{c++,thong,lomb,vanicek}:
\begin{equation}
\tan(4\pi f\tau)=\frac{\sum_{i=1}^{n}\sin(4\pi f t_{i})}{\sum_{j=1}^{n}\cos(4\pi f t_{j})}. \label{tau}
\end{equation}
gdzie $t_{i}$ to punkty ścieżki czasu szeregu $RR$ (sumy kumulatywnej szeregu $RR$, tj.
zbioru $\{t_0=RR_0, t_1=RR_0+RR_1, t_2=RR_0+RR_1+RR_2, \ldots, \sum_{i=0}^n RR_i \}$)
odpowiadające poszczególnym odstępom $RR$ w~następujący sposób
\begin{equation}
RR(t_{i})\equiv RR_{i}\equiv t_{i}-t_{i-1}, \label{RRy}
\end{equation}
czyli $t_{i}$ odpowiada pozycji kompleksu QRS o~numerze $i+1$.

Powyższa metoda jest szczególnie użyteczna jeśli stosujemy ją dla różnych szeregów
czasowych $RR$ pochodzących od różnych osób w celu wyodrębnienia wspólnych częstotliwości,
i nie jest, jak napisano wcześniej, stosowany resampling. W literaturze można znaleść m.in
następujące dwa opisy efektywności metod (LSP/ALSP): (a) w artykule \cite{laguna} autorzy 
stosują LSP wyznaczając widmo szeregu $RR$ ze szczególnym naciskiem na zmienną szerokość próbkowania oraz przerwy w sygnale wynikające z usuwania pobudzeń ektopowych; (b) w pracy
\cite{thong} stosują ALSP dla okien czasowych o stałej szerokości z tego
samego szeregu $RR$. Opisywana tutaj metoda ALSP dla różnych szeregów z różnych źródeł
jest technicznie bardzo podobna do tej z \cite{thong} jednakże ich własności matematyczne
oraz zakres interpretacji znacznie się różni. Pierwsza podstawowa różnica polega na sposobie
denormalizacji periodogramu w którym zamiast pełnej wariancji $SDNN^2$ jak w definicji
(\ref{eq:lomb}) stosuje się fragment pasma częstotliowościowego który można fizjologicznie
zinterpretować i wariancję oblicza się w zakresie 0.04 - 0.4 Hz, $SDNN^2$ zostaje zastąpione
przez poniższe wyrażenie:
\begin{equation}
  \sigma = \sum_{f\in[0.04,0.4Hz]}A_f^2,
\end{equation}
gdzie $A_f^2$ jest amplitudą sinusoidy o~częstotliwości $f$. Druga różnica to końcowe
uśrednianie po grupie szeregów pochodzących od różnych osób, czyli:
\begin{equation}
  \overline{P_{X}(f)} = \frac{1}{M}\sum_{k=1}^{M}P_{X}^k(f),
  \label{eq:alsp}
\end{equation}
gdzie $M$ jest liczbą analizowanych szeregów a $P_{X}^k(f)$ jest periodogramem
Lomba-Scargle'a dla $k$-tego szeregu. Podsumowując, uśredniony LSP można zinterpretować
jako średni wkład pasma częstotliwościowego w zadanym zakresie widma 0.04 - 0.4 Hz.
Wszystkie częstotliwości które przejawiają podobne zachowanie w sensie spektralnym w 
indywidualnych szeregach danej grupy badanych osób będą wzmacniane w wyniku uśredniania.
Podobieństwo zachowania spektralnego częstości polega na spójnie wysokim lub niskim
wkładzie w całkowitej wariancji lub mocy. Jeśli zachodzi sytuacja odwrotna, to znaczy wkład
częstości w całkowitej mocy ulega zmianie z nagrania na nagranie wtedy następuje wygaszanie
tychże częstości w wyniku uśrednienia.