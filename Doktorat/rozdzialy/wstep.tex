\chapter*{Wstęp}

Od dość dawna, na długo przed wynalezieniem elektrokardiografu lekarze zdawali sobie
sprawę ze znaczenia badań nad rytmem serca. Jednakże przez setki lat aż do ubiegłego
stulecia generalnie jedyną techniką badawczą było osłuchiwanie. To co się nasuwało
z tych wieloletnich badań to że zmiany w rytmie serca, z uderzenia na uderzenie,
są związane z różnymi czynnikami jak np. wiek, choroba, czy ogólna kondycja
psychofizyczna pacjenta. Warto zwrócić uwagę na ogromne znaczenie badań rytmu serca w
diagnostyce medycznej Chin, także starożytnych Chin. Jednakże dokładne pomiary rytmu
serca, podlagające analizie jakościowej i ilościowej z wykorzystaniem narzędzi
matematycznych, stały się dopiero możliwe wraz z rozwojem technologicznym poczynając
od galwanometru, kimografu, wariografu pisakowego, a kończąc na cyfrowym przetwarzaniu
sygnału.

Rozwój galwanometru w XIX wieku jest związany z takimi nazwiskami jak Luigi Galvani,
Allesandro Volta, Andre-Marie Ampere, Hans Christian Oersted. Galwanometr pozwalał na
pomiar bardzo niewielkich prądów w tym biopotencjał generowany przez serce.
W 1847 roku Ludwig wynalazł kimograf z bębnem z okopconego papieru który służył
do pomiaru mechanicznej aktywności jak np ciśnienie pulsu serca. Z kolei w 1894
MacKenzie wynalazł wariograf atramentowy, a inny wynalazca, Einthoven,
połączył działanie galwanometru z możliwością wykonywania zdjęć, dzięki czemu
można było uzyskać wykresy elektrycznej aktywności serca. Natomiast rozwój
elektrokardiografu umożliwił badanie prawidłowej i nieprawidłowej elektrycznej
przewodności mięśnia sercowego.

Pierwsze udokumentowane badanie zmienności rytmu serca, u koni, przypisuje się Hales'owi
w 1733 roku. W 1847 roku Ludwig, dzięki kimografowi mógł zaobserwować, związek
pomiędzy cyklem oddechowym a zmianami częstości pulsu u psa,
co może być uznane jako pierwsza obserwacja niemiarowości oddechowej rytmu zatokowego,
w skrócie RSA (ang - Respiratory Sinus Arrhythmia). W drugiej połowie XIX wieku
podawano dwie potencjalne przyczy RSA: pierwsza, w 1865, w której Traube postawił
hipotezę, ze komórki rdzenia kręgowego kontrolujące rytm serca mogą być kontrolowane
fazowo przez bezpośrednie oddziaływanie z rdzenia przedłużonego. Współczesne
neuroanatomiczne i neurofizjologiczne badania dostarczają argumentów za współczesną
wersją tego modelu, który opiera się na generacji podstawowej częstotliwości sercowo
naczyniowej przez miedzyneuronową sieć tworzoną przez komórki rdzenia kręgowego
oraz druga zaproponowana przez Hering'a w 1871 roku który zapostulował że mechanizmem
powodującym RSA może być odruchowa regulacja sercowo-naczyniowych centrów
regulacyjnych przez sprzężenie zwrotne po pobudzeniu włókien aferentnych przez
receptory płucne. Co ciekawe, wczesne wzmianki na temat RSA można znaleść w pracach z
dziedziny psychologii, choć nie były one tak rozumiane.

W wielu publikacjach na przestrzeni lat daje się zauważyć wyrażane przez różnych
autorów znaczenie zmienności rytmu serca jako wskaźnik fizjologiczny. Wczesne prace
koncentrowały się głównie na RSA i można powiedzieć, że nie było zbyt dużego
rozróżnienia pomiędzy RSA a arytmią zatokową. Badanie nad zmiennością rytmu serca
początkowo rozwijały się w dwóch kierunkach: pierwszy koncentrował się na
zrozumieniu fizjologicznych mechanizmów wpływających na zmienność rytmu serca,
drugi był próbą znalezienia związku pomiędzy zmiennością rytmu serca a stanem
klinicznym pacjenta, ponadto można wraz z rozwojem polygrafu i jego zastosowaniem w
badaniach akademickich w latach 60-tych XX wieku wyodrębnić trzeci trend jako
poszukiwanie związków pomiędzy fizjologicznymi procesami a zmiennością rytmu serca.

Związki pomiędzy fizjologicznymi procesami a zmiennością rytmu serca można znaleść
w pracy Bainbridg'a z 1920 roku będącej próbą wyrażenia RSA w duchu zmian w
baroreceptorach i receptorach pojemnościowych w związku ze zmianami ciśnienia płucnego.
Także inni badacze jak Anrep, Pascul i Rossler zajmowali się fizjologicznymi podstawami
RSA i ich wkład jest uznawany jako pierwsze systematyczne i dogłębne studium RSA.
Trzeba także wspomnieć o pracy Hering'a z 1910 roku opisującej relacje pomiędzy
wielkością amplitudy RSA a częstotliwością pobudzenia z nerwu błednego.

Prace Eppinger'a i Hess'a z 1915 roku zapoczątkowały kliniczny aspekt badań nad
zmiennością rytmu serca i koncentrowały się na klinicznych aspektach związanych z
nieprawidłowościami w działaniu funkcji autonomicznych. Ich badania były bardzo
istotne gdyż ujawniały ważną role wegetatywnego układu nerwowego w schorzeniach
klinicznych i sugerowały związki pomiędzy tymi schorzeniami a patologiami psychicznymi,
ponadto ujawniały wrażliwość nerwu błędnego na substancje o działaniu cholinergicznym.

Zależność pomiędzy zmiennością rytmu serca a stanem systemu nerwowego była
przedmiotem badań u Wolf'a, 1967 roku, i u Hon'a, lata 1958, 1963. Hon wskazywał że
specyficzne zmiany w zmienności rytmu serca mogą być przejawem stanu zagrożenia płodu.
Z kolei Wolf wskazywał na istotne znaczenie centralnego układu nerwowego w przypadku
nagłej śmierci o podłożu kardiologicznym, w tym obszarze zmienność rytmu serca była
miernikiem komunikacji pomiędzy mózgiem, sercem a nerwem błędnym.
Należy wspomnieć, że wczesne prace ujmowały rytm serca jako zmienną zależną od
procesów kognitywnych, np. Lacey w 1967 roku, czy też procesów metabolicznych np Obrist,
1981 roku. Czasami w tym okresie zmienność rytmu serca była postrzegana jako wariancja
błędu związanego z niezbyt dokładnym aparatem pomiarowym. 

Wzrastające z czasem znaczenie zmienności rytmu serca jako interesujące zjawisko samo
w sobie oraz użycie tego fenomenu jako zmienną opisową spowodowało wyodrębnienie jej
znaczenia od konkretnych zjawiskach fizjologicznych. Wcześniejsze badanie nad zmiennością
rytmu serca odpowiadały następującym koncepcjom: (a) model oparty na indywidualnych
różnicach, traktujący zmienność rytmu serca jako zmienną charakterystyczną,
która cechuja przewidywalne wzorce zachowania iautonomiczności (np. w pracach Lacey \&
Lacey, 1958; Porges, 1972; Price, 1975; Thackray, Jones, \& Touchstone, 1975),
(b) zmienność rytmu serca jako przejaw zdrowia psychicznego (np. w pracach:
 Kahnegman, 1973; Kalsbeek \& Ettema, 1963; Lacey, 1967; Porges \& Raskin, 1969;
 Sayers, 1973), (c) kontrola zmienności rytmu serca przez oddziaływanie technikami
 biologicznego sprzężenia zwrotnego (np. w pracach Hnatiow \& Lang, 1965; Lang, Sroufe,
 \& Hastings, 1967). Oczywiście granice pomiędzy wspomnianymi koncepcjami nie zawsze są
 ostre, co z jednej strony nie ułatwia pracy badaczom, z drugiej jest ciekawym naukowym
wyzwaniem. W bardziej współczesnych badaniach wzrasta świadomość że jakość badań
ilościowych i prawidłowej interpretacji zmienności rytmu serca jest zależna nie tylko
od zrozumienia fizjologicznych procesów będących jego źródłem ale także od
związków pomiędzy tymi procesami a procesami behawioralnymi.

Historycznie można wyróżnić dwa powiązane między sobą podejścia w celu pomiaru i
ilościowej analizy zmienności rytmu serca. We wcześniejszych, szczególnie użytecznych
do badania zmienności rytmu serca płodu, stosowano krótko-okresowe tachogramy oraz
używano prostych estymatorów numerycznych np. różnica pomiędzy najkrótszym i
najdłuższym cyklem uderzeń serca. Nowsze podejścia do analizy wykorzystują bardziej
wyrafinowany aparat matematyczny - dystrybucje statystyczne (np. prace: Kleiger, Miller,
Bigger, Moss, \& Multicenter Post-Ifraction Research Group, 1987). Nowoczesne statystyczne
podejście traktuje zbiór interwałów R-R lub pary takich interwałów jako czasowo
nieuporządkowane dane i wyraża ich zmienność poprzez standardowe pojęcia statystyczne
albo poprzez geometryczne właściwości histogramów lub innych reprezentacji
geometrycznych (np Malik, 1995)

Analiza jakościowa ujawniła jedną z cech zmienności rytmu serca, mianowicie specyficzne
wzorce jakie można zaobserwować w zmienności rytmu serca daje się przyporządkować
pewnym konkretnym fizjologicznym procesom. Nowoczesne analityczne metody mające za dane
wejściowe serie interwałów R-R umożliwiły uzyskanie okresowych składowych ze wzorców
zmian w cyklach uderzeń serca oraz umożliwiły bardziej kompletną i wyrafinową
reprezentację danych, przy okazji umożliwiając zmiany w teoretycznym matematycznym
opisie tych danych. Tacy badacze jak Chess, Tam, Calaresu (1975) i Sayers (1973) do analizy
ilościowej szeregów czasowych R-R używali analizy spektralnej. Natomiast Porges i inni
do opisu związków pomiedzy zmiennością rytmu serca a oddychaniem stosowali analizę
spektrum krzyżowego i wynikiem jej zastosowania był wniosek że gestości spektralne
związane z widmem uderzeń serca mogą wyrażać oddziaływanie nerwu błędngo na serce.
Natomiast Akselrod i inni, w 1981 roku, zaproponowali następującą interpretacje
wyników analizy statystycznej: składowa rytmu oddechowego w spektrum częstotliwości
rytmu serca odpowiada oddziaływaniu nerwu błednego natomiast dwie niższe
częstotliwości reprezentują oddziaływanie pomiędzy nerwem błędnym a współczulnym
układem nerwowym. W latach 70-tych XX wieku Ewing i inni opierając się na
krótkoterminowych szeregach czasowych R-R byli w stanie określić autonomiczną
neuropatię wśród pacjentów z cukrzycą. Badania zmienności rytmu serca w późnych
latach 80-tych XX wieku także przyniosły potwierdzenie ich wartości klinicznej gdyż
stały się wyraźnym predykatorem umieralności po ostrym zawale serca. Wraz z pojawieniem
się 24 godzinnych zapisów EKG, czyli możliwością analizy długoterminowych szeregów
czasowych R-R, badacze uzyskali dodatkowe dane pomocne do określenia stratyfikacji ryzyka,
czy też ogólnie do badań stanu fizjologicznego lub patologii u pacjentów z problemami
kardiologicznymi.

Wcześniejsze podejście statystyczne do analizy zmienności rytmu serca korzystało
głównie ze statystyki opisowej jak np w pracach Lacey \& Lacey, 1958; Lang i inni, 1967;
Porges \& Raskin, 1969. Już nowsze badania datowane na lata 70-te XX wieku kładą
większy nacisk na analizę szeregów czasowych przy badaniu zmienności rytmu serca.
Zostosowanie szeregów czasowych okazało sie zasadne z dwóch powodów: pierwszy, szeregi
czasowe umożliwiły uzyskanie informacji na temat periodycznych składowych w rytmie
serca, nie bez znaczenia oraz sporej użyteczności w pracy klinicznej, drugi umożliwiły
uzyskanie wielu innych cennych informacji w dziedzinie fizjologii a także ułatwiły
modelowanie związków pomiędzy fizjologicznymi i psychicznymi procesami.
