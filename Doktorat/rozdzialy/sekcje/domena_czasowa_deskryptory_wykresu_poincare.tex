\subsection{Deskryptory wykresu Poincar\ee}


Następna techniką z domeny czasowej wykorzystywaną do analizy HRV to technika opierająca
się na tzw. wykresach Poincar\ee.

Wykres Poincar\ee, którego nazwa pochodzi od twórcy tej techniki Henri Poincare, jest 
analityczno-wizualną technika ułatwiającą analizę złożonych zjawisk i ma ona zastosowanie
nie tylko w naukach medycznych ale także m.in. w meteorologii, geofizyce, astronomii.
Element wizualny wynika z użycia odpowiednio przygotowanych wykresów i w praktyce jest to
bardzo ważny element gdyż umożliwia na bardzo szybką wstępna jakościową analizę danego
zjawiska. Element analityczny jest związany z parametrami, tak zwanymi deskryptorami,
które wyrażają ilościowo informacje zawarte w wykresie Poincar\ee. Należy zaznaczyć że
technika ta jest bardzo odporna na dane zawierające artefakty lub wartości odstające i w
tym sensie ma przewagę nad technikami które wymagają niezakłóconych danych np 
wykorzystujące szybką transformatę Fouriera (FFT).

Konstrukcja wykresów Poincare\ee  jest dość prosta, przyjmując że szereg $RR$ wyrazimy jako
wektor:
\begin{equation}
\bRR = (RR_{1}, RR_{2}, \ldots, RR_{n})
\end{equation}
oś odciętych jest utworzona z wektora $RR$ bez ostatniego elementu, natomiast oś rzędnych
zawiera elementy bezpośrednio po nich następujące, czyli wektor $RR$ bez elementu
pierwszego:
\begin{equation}
\bRRm = (RR_{1}, RR_{2}, \ldots, RR_{n-1}),
\bRRn = (RR_{2}, RR_{2}, \ldots, RR_{n}) 
\end{equation}
zatem wykres Poincar\ee jest chmurą punktów $(RR_{i}, RR_{i+1})$ gdzie $i=(1, \ldots, n-1)$
Podstawowymi parametrami (deskryptorami) utworzonymi w oparciu o powyższą reprezentacją 
graficzną wektora $\bRR$ są parametry określane w literaturze symbolami $SD1$ i $SD2$, które
wyrażają odpowiednio zmienność krótko- i długoterminową rytmu serca,
matematycznie wyrażonymi przy pomocy wariancji:

\begin{equation}
SD1^2 = Var\left(\frac{\bRRn - \bRRm}{\sqrt{2}}\right)
\end{equation}

\begin{equation}
SD2^2 = Var\left(\frac{\bRRn + \bRRm}{\sqrt{2}}\right)
\end{equation}

Tak zdefiniowane jak powyżej parametry $SD1$, $SD2$ mierzą rozproszenie punktów wykresu
Poincar\ee względem odpowiednio tzw linii identyczności $l_{I}$ (linii dla której $\bRRm = \bRRn$)
oraz linii prostopadłej do linii identyczności $l_2$ przechodzącej przez centroid. Na
poniższym rysunku została przedstawiona konstrukcja tych parametrów oraz zostały
przedstawione rzuty wariancji: histogram (a) krótkoterminowej zmienności rytmu serca $SD1$
wzdłuż linii identyczności $l_{I}$, histogram (b) pełnej zmienności rytmu serca na oś $RR_n$,
histogram (c) długoterminowej zmienności rytmu serca $SD2$ wzdłuż linii prostopadłej $l_2$
do linii identyczności $l_{I}$.

\begin{figure}
\centering
\includegraphics[width=\textwidth]{graph/pp_distrib.jpg}
\caption{Rysunek prezentuje położenie prostej równoległej ($l_1$) i~prostej prostopodłej ($l_2$) do osi identyczności separującej punkty reprezentujące skrócenia i~wydłużenia odstępów $RR$. Na wykresie przedstawiono również rozkłady punktów opisujących zmienność krótkoterminową $SD1$ (panel $a$ - rzut PP na oś $l_2$), zmienność długoterminową $SD2$ (panel $c$ reprezentujący rzut PP na oś $l_1$) oraz zmieność całkowitą (panel $b$ -- rzut PP na oś odciętych). Na histogramach różnymi odcieniami szarości oznaczono wkłady pochodzące od punktów, znajdujących się nad i~pod osią identyczności. Opracowano na podstawie \cite{hrstruct} -- rysunek udostępniony przez Jarosława Piskorskiego i~Przemysława Guzika na licencji CC BY.}
\label{fig:pp_distrib}
\end{figure}
  
RYSUNEK


Wariancyjne deskryptory $SD1$ i $SD2$, możemy wyrazić przy pomocy następujących wzorów: 

\begin{equation}
SD1^2 = \frac{1}{n}\sum_{i=1}^{n}r_{i}^{\perp 2}
\end{equation}

\begin{equation}
SD2^2 = \frac{1}{n}\sum_{i=1}^{n}r_{i}^{\parallel 2}
\end{equation}

gdzie $r_{i}^{\perp}$ jest prostopadłą odległością punktu $(RR_{i}, RR_{i+1})$ do linii
identyczności, $r_{i}^{\parallel}$ jest odległością punktu $(RR_{i}, RR_{i+1})$ do linii
$l_2$ wzdłuż linii identyczności.
Należy zaznaczyć że wyrażenie () dla krótkoterminowej zmienności rytmu serca
nie jest z matematycznego punktu widzenia odchyleniem standardowym, gdyż nie jest
wyznaczane względem linii $l_1$ przechodzącej przez centroid, to znaczy nie minimalizuje
drugiego momentu rozkładu punktów $RR$. Jednakże to w żaden sposób
nie umniejsza użyteczności tego deskryptora i to z dwóch powodów: (1) linia
identyczności dzieli cały wykres Poincar\ee na dwie części, górną dotyczącą zwolnień,
dolną dotyczącą przyspieszeń, zatem $SD1$ liczona względem $l_{I}$ ma wyraźną interpretację
fizjologiczną, (2) różnica pomiędzy wariancjami odnoszącymi się odpowiednio do linii $l_1$
oraz $l_{I}$ zgodnie z wartościami podanymi w 
(uzupelnic odnośnik do bibliografii - 'Geometry of \ldots  and it's asymmetry in healthy adults 2007, ')
jest rzędu $10^{-5}$ i dąży do $0$ dla coraz większej liczby pomiarów, ponadto sam błąd
związany z wprowadzeniem wariancji względem $l_{I}$ jest rzędu $10^{-8}$, zatem w praktyce
jest zaniedbywalny.

Relacja pomiędzy krótkoterminową i długoterminową zmiennością rytmu serca opisanymi powyżej
a miarą pełnej zmienności rytmu serca opisanej w (ODNOSNIK DO ODPOWIEDNIEGO ROZDZIALU)
jest następująca:
\begin{equation}
SDNN^2 = \frac{1}{2}(SD1^2 + SD2^2)
\end{equation}