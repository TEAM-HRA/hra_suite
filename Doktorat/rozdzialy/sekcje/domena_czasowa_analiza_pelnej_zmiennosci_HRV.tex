\subsection{Analiza pełnej zmienności HRV}

Jednym ze sposobów analizy zmienności rytmu serca są metody z zakresu domeny czasowej,
które były stosowane historycznie najwcześniej. Wśrod nich można wyróżnić metody
badające cały szereg odstępów $RR$, powszechnie w literaturze przedmiotu oznaczającym
odległości $R$ lub inaczej czas pomiędzy kolejnymi załamkami w elektrokardiogramie.

Ze względu na użyty aparat matematyczny można wyróżnić tzw podejście wariancyjne w którym
powszechnie stosowanym parametrem jest parameter oznaczany jako $SDNN$ będący miarą pełnej
zmienności rytmu serca a wyrażony następującym wzorem:  

\begin{equation}\label{eq6}
SDNN = \sqrt{\frac{1}{n}\sum_{i=1}^{n}(RR_{i} - \overline{RR})^{2}}
\end{equation}

czyli jako pierwiastek z pełnej wariancji szeregu $RR$, gdzie $n$ jest to liczba odstępów $RR$
w całym nagraniu, $\overline{RR}$ oznacza zwykłą średnią, czyli:

\begin{equation}
\overline{RR} = \sum_{i=1}^{n}RR_{i}
\end{equation}

Nietrudno jest zauważyć że powyższe wyrażania odpowiadają drugiemu i pierwszemu momentowi
rozkładu dla szeregu odstępów $RR$. Sam w sobie szereg, wynika to także z natury
zjawiska jakim jest rytm serca, jest szeregiem niestacjonarnym w mocnym jak i słabym
sensie. Ten fakt oznacza że średnia jak i wariancja fluktuuje w czasie lub inaczej zmienia
się w zależności od tego który odcinek szeregu $RR$ analizujemy. W praktyce najcześciej
rozpatrywanymi długościami odcinka jest 5 lub 30 minut lub nagrania 24 godzinne.
