\subsection{Periodogram Lomba}

Wśród metod parametrycznych możemy wyróżnić metodę będącą rozwinięciem analizy spektralnej,
znaną pod nazwą periodogramu Lomba-Scargle'a (LSP). Szczególna użyteczność tej metody
polega między innymi na tym że nie wymaga resamplingu, tak jak wcześniej wspomniana 
metoda autoregresyjna. Jest ona realizowana przy pomocy metody najmniejszych kwadratów.
Ze względu na brak resamplingu nie jest konieczne aby szereg czasowy, np. $RR$, miał stałą
częstość próbkowania, zatem można użyć bezpośrednio oryginalnych danych \cite{scargle, lomb_o, c++, thong},
a wszelkie pobudzenia ektopowe, zwane inaczej pobudzeniami przeniesionymi oraz
artefakty techniczne nie wpływają na poprawność działania periodogramu Lomba-Scargle'a.

Pochodna powyższej metody o nazwie uśrednionego periodogramu Lomba-Scargle'a została
opracowana w trakcie eksperymentu w którym osoby biorące w nim udział miały za zadanie
oddychać z zadaną częstotliwością. W eksperymencie badacze weryfikowali
hipotezę czy istnieje sprzężenie mechaniczne pomiędzy czynnością układu sercowo-naczyniowego
a oddechem o określonym rytmie. 

Technicznie metoda uśrednionego periodogramu Lomba-Scargle'a (ALSP) wykorzystuje uśredniony i
znormalizowany periodogram Lomba-Scargle'a dla badanej populacji szeregów. Wyrażanie
definiujące znormalizowany periodogram z odjętą średnią dla jednego szeregu z populacji
jest następujące \cite{c++, thong, vanicek}:

\begin{eqnarray}
P_{X}(f)&=&\frac{1}{2SDNN^{2}}\left\{\frac{\left[ \sum_{i=1}^{n}(RR(t_{i})-\overline{RR})\cos(2\pi f(t_{i}-\tau))\right]^{2}}{\sum_{i=1}^{n}\cos^{2}(2\pi f(t_{i}-\tau))}\right. \label{LSPeriodogram}\nonumber\\ 
&+&\left. \frac{\left[ \sum_{i=1}^{n}(RR(t_{i})-\overline{RR})\sin(2\pi f(t_{i}-\tau))\right]^{2}}{\sum_{i=1}^{n}\sin^{2}(2\pi f(t_{i}-\tau))}\right\}
\label{eq:lomb} ,
\end{eqnarray}
gdzie $n$ jest długością szeregu czasowego, wyrażoną w~punktach, a~nie w~czasie,
$\overline{RR}$ jest jego średnią, $SDNN^{2}$ jego wariancją, a~$\tau(t)$ jest zależnym od
częstości przesunięciem fazowym sprawiającym, że periodogram jest niezmienniczy względem
przesunięć w czasie \cite{c++,thong,lomb,vanicek}:
\begin{equation}
\tan(4\pi f\tau)=\frac{\sum_{i=1}^{n}\sin(4\pi f t_{i})}{\sum_{j=1}^{n}\cos(4\pi f t_{j})}. \label{tau}
\end{equation}
gdzie $t_{i}$ to punkty ścieżki czasu szeregu $RR$ (sumy kumulatywnej szeregu $RR$, tj.
zbioru $\{t_0=RR_0, t_1=RR_0+RR_1, t_2=RR_0+RR_1+RR_2, \ldots, \sum_{i=0}^n RR_i \}$)
odpowiadające poszczególnym odstępom $RR$ w~następujący sposób
\begin{equation}
RR(t_{i})\equiv RR_{i}\equiv t_{i}-t_{i-1}, \label{RRy}
\end{equation}
czyli $t_{i}$ odpowiada pozycji kompleksu QRS o~numerze $i+1$.

Powyższa metoda jest szczególnie użyteczna jeśli stosujemy ją dla różnych szeregów
czasowych $RR$ pochodzących od różnych osób w celu wyodrębnienia wspólnych częstotliwości,
i nie jest, jak napisano wcześniej, stosowany resampling. W literaturze można znaleść m.in
następujące dwa opisy efektywności metod (LSP/ALSP): (a) w artykule \cite{laguna} autorzy 
stosują LSP wyznaczając widmo szeregu $RR$ ze szczególnym naciskiem na zmienną szerokość próbkowania oraz przerwy w sygnale wynikające z usuwania pobudzeń ektopowych; (b) w pracy
\cite{thong} stosują ALSP dla okien czasowych o stałej szerokości z tego
samego szeregu $RR$. Opisywana tutaj metoda ALSP dla różnych szeregów z różnych źródeł
jest technicznie bardzo podobna do tej z \cite{thong} jednakże ich własności matematyczne
oraz zakres interpretacji znacznie się różni. Pierwsza podstawowa różnica polega na sposobie
denormalizacji periodogramu w którym zamiast pełnej wariancji $SDNN^2$ jak w definicji
(\ref{eq:lomb}) stosuje się fragment pasma częstotliowościowego który można fizjologicznie
zinterpretować i wariancję oblicza się w zakresie 0.04 - 0.4 Hz, $SDNN^2$ zostaje zastąpione
przez poniższe wyrażenie:
\begin{equation}
  \sigma = \sum_{f\in[0.04,0.4Hz]}A_f^2,
\end{equation}
gdzie $A_f^2$ jest amplitudą sinusoidy o~częstotliwości $f$. Druga różnica to końcowe
uśrednianie po grupie szeregów pochodzących od różnych osób, czyli:
\begin{equation}
  \overline{P_{X}(f)} = \frac{1}{M}\sum_{k=1}^{M}P_{X}^k(f),
  \label{eq:alsp}
\end{equation}
gdzie $M$ jest liczbą analizowanych szeregów a $P_{X}^k(f)$ jest periodogramem
Lomba-Scargle'a dla $k$-tego szeregu. Podsumowując, uśredniony LSP można zinterpretować
jako średni wkład pasma częstotliwościowego w zadanym zakresie widma 0.04 - 0.4 Hz.
Wszystkie częstotliwości które przejawiają podobne zachowanie w sensie spektralnym w 
indywidualnych szeregach danej grupy badanych osób będą wzmacniane w wyniku uśredniania.
Podobieństwo zachowania spektralnego częstości polega na spójnie wysokim lub niskim
wkładzie w całkowitej wariancji lub mocy. Jeśli zachodzi sytuacja odwrotna, to znaczy wkład
częstości w całkowitej mocy ulega zmianie z nagrania na nagranie wtedy następuje wygaszanie
tychże częstości w wyniku uśrednienia.