\subsection{Transformata Fouriera}

W ogólnym zarysie transformata Foriera polega na rozkładzie sygnału na szereg funkcji
sinusidalnych z odpowiednio dobranymi aplitudami i częstotliwościami. Istnieją cztery typy
transformat \cite{dsp} które dzielą się ze względu na: (a) zachowanie sygnału - periodyczny,
aperiodyczny; (b) typ sygnału - ciągły, dyskretny. Wspólną ich cechą jest to że
zawsze operują na sygnałach nieskończonych, zatem sygnał uzyskany z ograniczonego w czasie
szeregu $RR$ nie jest poprawnym, z formalnego punktu widzenia, źródłem danych dla
transformaty Fouriera. Najcześciej tą niedogodność rozwiązuje się na dwa sposoby:
(a) przyjmuje się wartości zero z lewej i prawej strony sygnału, zatem uzyskuje się sygnał
dyskretny aperiodyczny; lub (b) zakłada się występowania kopii sygnału z lewej i prawej jego
strony, zatem uzyskuje się sygnał dyskretny periodyczny. Z powodów praktycznych
lepsze do obliczeń numerycznych jest podejścia (b) gdyż nie ma problemu z koniecznością
reprezentowania aperiodycznego sygnału przez nieskończoną liczbę sinusoid, czyli w tym
przypadku stosuje się tzw dyskretną transformatę Fouriera (ang. DFT - Discrete Fourier
Transform). Działanie DFT na $N$-elementowy dyskretny sygnał czasowy generuje dwa $N/2+1$
elementowe szeregi amplitud $C$ oraz $S$ dla funkcji \emph{cosinus} oraz \emph{sinus}, co
wyraża następująca relacja:

\begin{equation}
  x[i] = \sum_{k=0}^{N/2} \bar{C}[k]\cos( \frac{2\pi k i}{N}) + \sum_{k=0}^{N/2}\bar{S}[k]\sin( \frac{2\pi k i}{N})
  \label{eq:dft}
\end{equation}
gdzie $x$ to szereg z domeny czasowej, $\bar{C}[k] = \frac{C[k]}{N/2}$, $\bar{S}[k] = -\frac{S[k]}{N/2}$, $\bar{C}[0] = \frac{C[0]}{N}$, $\bar{C}[N/2] = \frac{C[N/2]}{N}$ a $i=[0,\ldots,N]$. Indeks $k$ reprezentuje
częstotliwości funkcji \emph{cosinus} oraz \emph{sinus}, a $C[k]$ i $S[k]$ opisują
amplitudę, przypadającą na daną częstotliwość. Postać wzoru (\ref{eq:dft}) określa także
sposób przejścia (syntezy) sygnału z domeny częstotliwości do domeny czasowej.
Jadnakże należy zwrócić uwagę na bardzo ważny fakt związany z analizą szeregów czasowych
$RR$. Pełna transformata Fouriera zawiera dwa człony:
amplitudowe widmo mocy które odpowiada wariancji z domeny czasowej oraz widmo fazowe
które nie jest rozpatrywane w obszarze HRV. Fakt odrzucania członu fazowego powoduje że
nie można przejść z domeny częstotliwości do domeny czasowej i dotyczy to specyficznie
szeregów czasowych $RR$, a nie ogólnych matematycznych rozważań w kontekście relacji (\ref{eq:dft}). 
Powyższe zdanie można wyrazić obrazowo w ten sposób: nie jest możliwe odtworzenie kształtu
szeregu $RR$, w szczególności serii zwolnień i przyspieszeń, wychodząc z domeny
częstotliwości do domeny czasowej z powodu 'zgubienia' widma fazowego w transformacie
Fouriera.

Użycie DFT do analizy spektralnej HRV, wymaga aby wejściowy sygnał bedący szeregiem odstępów
$RR$ przedstawić w postaci tachogramu o równoodległych punktach \cite{splines}. Tachogram jest następnie
interpolowany oraz wygładzany przy pomocy odpowiedniego filtra np typu \emph{boxcar} \cite{boxcar}.

Spośród kilku metod przy pomocy których można wykonać analizę spektralną DFT,
algorytm szybkiej transformaty Fouriera (ang. fast Fourier transform, FFT) \cite{cooleyfft} cieszy się
największym uznaniem z powodu najlepszej wydajności obliczeniowej.